\chapter{The way of the program}

The goal of this book is to teach you to think like a computer
scientist.  This way of thinking combines some of the best features of
mathematics, engineering, and natural science.  Like mathematicians,
computer scientists use formal languages to denote ideas (specifically
computations).  Like engineers, they design things, assembling
components into systems and evaluating tradeoffs among alternatives.
Like scientists, they observe the behavior of complex systems, form
hypotheses, and test predictions.  \index{problem solving}

The single most important skill for a computer scientist is {\bf
  problem solving}.  Problem solving means the ability to formulate
problems, think creatively about solutions, and express a solution
clearly and accurately.  As it turns out, the process of learning to
program is an excellent opportunity to practice problem-solving
skills.  That's why this chapter is called, ``The way of the
program''.

On one level, you will be learning to program, a useful skill by
itself.  On another level, you will use programming as a means to an
end.  As we go along, that end will become clearer.


\section{What is a program?}

A {\bf program} is a sequence of instructions that specifies how to
perform a computation.  The computation might be something
mathematical, such as solving a system of equations or finding the
roots of a polynomial, but it can also be a symbolic computation, such
as searching and replacing text in a document or something
graphical, like processing an image or playing a video.
\index{program}

The details look different in different languages, but a few basic
instructions appear in just about every language:

\begin{description}

\item[input:] Get data from the keyboard, a file, the network, or some
other device.

\item[output:] Display data on the screen, save it in a
file, send it over the network, etc.

\item[math:] Perform basic mathematical operations like addition and
multiplication.

\item[conditional execution:] Check for certain conditions and
run the appropriate code.

\item[repetition:] Perform some action repeatedly, usually with
some variation.

\end{description}

Believe it or not, that's pretty much all there is to it.  Every
program you've ever used, no matter how complicated, is made up of
instructions that look pretty much like these.  So you can think of
programming as the process of breaking a large, complex task
into smaller and smaller subtasks until the subtasks are
simple enough to be performed with one of these basic instructions.


\section{Running Python}

One of the challenges of getting started with Python is that you
might have to install Python and related software on your computer.
If you are familiar with your operating system, and especially
if you are comfortable with the command-line interface, you will
have no trouble installing Python.  But for beginners, it can be
painful to learn about system administration and programming at the
same time.
\index{running Python}
\index{Python!running}

To avoid that problem, I recommend that you start out running Python
in a browser.  Later, when you are comfortable with Python, I'll
make suggestions for installing Python on your computer.
\index{Python in a browser}

There are a number of web pages you can use to run Python.  If you
already have a favorite, go ahead and use it.  Otherwise I recommend
PythonAnywhere.  I provide detailed instructions for getting started
at \url{http://tinyurl.com/thinkpython2e}.  
\index{PythonAnywhere}

There are two versions of Python, called Python 2 and Python 3.
They are very similar, so if you learn one, it is easy to switch
to the other.  In fact, there are only a few differences you will
encounter as a beginner.
This book is written for Python 3, but I include some notes
about Python 2.
\index{Python 2}

The Python {\bf interpreter} is a program that reads and executes
Python code.  Depending on your environment, you might start the
interpreter by clicking on an icon, or by typing {\tt python} on
a command line. 
When it starts, you should see output like this:
\index{interpreter}

\begin{verbatim}
Python 3.4.0 (default, Jun 19 2015, 14:20:21) 
[GCC 4.8.2] on linux
Type "help", "copyright", "credits" or "license" for more information.
>>> 
\end{verbatim}
%
The first three lines contain information about the interpreter
and the operating system it's running on, so it might be different for
you.  But you should check that the version number, which is
{\tt 3.4.0} in this example, begins with 3, which indicates that
you are running Python 3.  If it begins with 2, you are running
(you guessed it) Python 2.

The last line is a {\bf prompt} that indicates that the interpreter is
ready for you to enter code.
If you type a line of code and hit Enter, the interpreter displays the
result: 
\index{prompt}

\begin{verbatim}
>>> 1 + 1
2
\end{verbatim}
%
Now you're ready to get started.
From here on, I assume that you know how to start the Python
interpreter and run code.


\section{The first program}
\label{hello}
\index{Hello, World}

Traditionally, the first program you write in a new language
is called ``Hello, World!'' because all it does is display the
words ``Hello, World!''.  In Python, it looks like this:

\begin{verbatim}
>>> print('Hello, World!')
\end{verbatim}
%
This is an example of a {\bf print statement}, although it
doesn't actually print anything on paper.  It displays a result on the
screen.  In this case, the result is the words

\begin{verbatim}
Hello, World!
\end{verbatim}
%
The quotation marks in the program mark the beginning and end
of the text to be displayed; they don't appear in the result.
\index{quotation mark}
\index{print statement}
\index{statement!print}

The parentheses indicate that {\tt print} is a function.  We'll get
to functions in Chapter~\ref{funcchap}.
\index{function} \index{print function}

In Python 2, the print statement is slightly different; it is not
a function, so it doesn't use parentheses.
\index{Python 2}

\begin{verbatim}
>>> print 'Hello, World!'
\end{verbatim}
%
This distinction will make more sense soon, but that's enough to
get started.


\section{Arithmetic operators}
\index{operator!arithmetic}
\index{arithmetic operator}

After ``Hello, World'', the next step is arithmetic.  Python provides
{\bf operators}, which are special symbols that represent computations
like addition and multiplication.  

The operators {\tt +}, {\tt -}, and {\tt *} perform addition,
subtraction, and multiplication, as in the following examples:

\begin{verbatim}
>>> 40 + 2
42
>>> 43 - 1
42
>>> 6 * 7
42
\end{verbatim}
%
The operator {\tt /} performs division:

\begin{verbatim}
>>> 84 / 2
42.0
\end{verbatim}
%
You might wonder why the result is {\tt 42.0} instead of {\tt 42}.
I'll explain in the next section.

Finally, the operator {\tt **} performs exponentiation; that is,
it raises a number to a power:

\begin{verbatim}
>>> 6**2 + 6
42
\end{verbatim}
%
In some other languages, \verb"^" is used for exponentiation, but
in Python it is a bitwise operator called XOR.  If you are not
familiar with bitwise operators, the result will surprise you:

\begin{verbatim}
>>> 6 ^ 2
4
\end{verbatim}
%
I won't cover
bitwise operators in this book, but you can read about
them at \url{http://wiki.python.org/moin/BitwiseOperators}.
\index{bitwise operator}
\index{operator!bitwise}


\section{Values and types}
\index{value}
\index{type}
\index{string}

A {\bf value} is one of the basic things a program works with, like a
letter or a number.  Some values we have seen so far are {\tt 2},
{\tt 42.0}, and \verb"'Hello, World!'".

These values belong to different {\bf types}:
{\tt 2} is an {\bf integer}, {\tt 42.0} is a {\bf floating-point number},
and \verb"'Hello, World!'" is a {\bf string},
so-called because the letters it contains are strung together.
\index{integer}
\index{floating-point}

If you are not sure what type a value has, the interpreter can
tell you:

\begin{verbatim}
>>> type(2)
<class 'int'>
>>> type(42.0)
<class 'float'>
>>> type('Hello, World!')
<class 'str'>
\end{verbatim}
%
In these results, the word ``class'' is used in the sense of
a category; a type is a category of values.
\index{class}

Not surprisingly, integers belong to the type {\tt int},
strings belong to {\tt str} and floating-point
numbers belong to {\tt float}.  
\index{type}
\index{string type}
\index{type!str}
\index{int type}
\index{type!int}
\index{float type}
\index{type!float}

What about values like \verb"'2'" and \verb"'42.0'"?
They look like numbers, but they are in quotation marks like
strings.
\index{quotation mark}

\begin{verbatim}
>>> type('2')
<class 'str'>
>>> type('42.0')
<class 'str'>
\end{verbatim}
%
They're strings.

When you type a large integer, you might be tempted to use commas
between groups of digits, as in {\tt 1,000,000}.  This is not a
legal {\em integer} in Python, but it is legal:

\begin{verbatim}
>>> 1,000,000
(1, 0, 0)
\end{verbatim}
%
That's not what we expected at all!  Python interprets {\tt
  1,000,000} as a comma-separated sequence of integers.  We'll learn
more about this kind of sequence later.
\index{sequence}

%This is the first example we have seen of a semantic error: the code
%runs without producing an error message, but it doesn't do the
%``right'' thing.
%\index{semantic error}
%\index{error!semantic}
%\index{error message}
% TODO: use this as an example of a semantic error later



\section{Formal and natural languages}
\index{formal language}
\index{natural language}
\index{language!formal}
\index{language!natural}

{\bf Natural languages} are the languages people speak,
such as English, Spanish, and French.  They were not designed
by people (although people try to impose some order on them);
they evolved naturally.

{\bf Formal languages} are languages that are designed by people for
specific applications.  For example, the notation that mathematicians
use is a formal language that is particularly good at denoting
relationships among numbers and symbols.  Chemists use a formal
language to represent the chemical structure of molecules.  And
most importantly:

\begin{quote}
{\bf Programming languages are formal languages that have been
designed to express computations.}
\end{quote}

Formal languages tend to have strict {\bf syntax} rules that
govern the structure of statements.
For example, in mathematics the statement
$3 + 3 = 6$ has correct syntax, but
$3 + = 3 \$ 6$ does not.  In chemistry
$H_2O$ is a syntactically correct formula, but $_2Zz$ is not.
\index{syntax}

Syntax rules come in two flavors, pertaining to {\bf tokens} and
structure.  Tokens are the basic elements of the language, such as
words, numbers, and chemical elements.  One of the problems with
$3 += 3 \$ 6$ is that \( \$ \) is not a legal token in mathematics
(at least as far as I know).  Similarly, $_2Zz$ is not legal because
there is no element with the abbreviation $Zz$.
\index{token}
\index{structure}

The second type of syntax rule pertains to the way tokens are
combined.  The equation $3 += 3$ is illegal because even though $+$
and $=$ are legal tokens, you can't have one right after the other.
Similarly, in a chemical formula the subscript comes after the element
name, not before.

This is @ well-structured Engli\$h
sentence with invalid t*kens in it.  This sentence all valid tokens
has, but invalid structure with.

When you read a sentence in English or a statement in a formal
language, you have to figure out the structure
(although in a natural language you do this subconsciously).  This
process is called {\bf parsing}.
\index{parse}

Although formal and natural languages have many features in
common---tokens, structure, and syntax---there are some
differences:
\index{ambiguity}
\index{redundancy}
\index{literalness}

\begin{description}

\item[ambiguity:] Natural languages are full of ambiguity, which
people deal with by using contextual clues and other information.
Formal languages are designed to be nearly or completely unambiguous,
which means that any statement has exactly one meaning,
regardless of context.

\item[redundancy:] In order to make up for ambiguity and reduce
misunderstandings, natural languages employ lots of
redundancy.  As a result, they are often verbose.  Formal languages
are less redundant and more concise.

\item[literalness:] Natural languages are full of idiom and metaphor.
If I say, ``The penny dropped'', there is probably no penny and
nothing dropping (this idiom means that someone understood something
after a period of confusion).  Formal languages
mean exactly what they say.

\end{description}

Because we all grow up speaking natural languages, it is sometimes
hard to adjust to formal languages.  The difference between formal and
natural language is like the difference between poetry and prose, but
more so: \index{poetry} \index{prose}

\begin{description}

\item[Poetry:] Words are used for their sounds as well as for
their meaning, and the whole poem together creates an effect or
emotional response.  Ambiguity is not only common but often
deliberate.

\item[Prose:] The literal meaning of words is more important,
and the structure contributes more meaning.  Prose is more amenable to
analysis than poetry but still often ambiguous.

\item[Programs:] The meaning of a computer program is unambiguous
and literal, and can be understood entirely by analysis of the
tokens and structure.

\end{description}

Formal languages are more dense
than natural languages, so it takes longer to read them.  Also, the
structure is important, so it is not always best to read
from top to bottom, left to right.  Instead, learn to parse the
program in your head, identifying the tokens and interpreting the
structure.  Finally, the details matter.  Small errors in
spelling and punctuation, which you can get away
with in natural languages, can make a big difference in a formal
language.


\section{Debugging}
\index{debugging}

Programmers make mistakes.  For whimsical reasons, programming errors
are called {\bf bugs} and the process of tracking them down is called
{\bf debugging}.
\index{debugging}
\index{bug}

Programming, and especially debugging, sometimes brings out strong
emotions.  If you are struggling with a difficult bug, you might 
feel angry, despondent, or embarrassed.

There is evidence that people naturally respond to computers as if
they were people.  When they work well, we think
of them as teammates, and when they are obstinate or rude, we
respond to them the same way we respond to rude,
obstinate people (Reeves and Nass, {\it The Media
    Equation: How People Treat Computers, Television, and New Media
    Like Real People and Places}).
\index{debugging!emotional response}
\index{emotional debugging}

Preparing for these reactions might help you deal with them.
One approach is to think of the computer as an employee with
certain strengths, like speed and precision, and
particular weaknesses, like lack of empathy and inability
to grasp the big picture.

Your job is to be a good manager: find ways to take advantage
of the strengths and mitigate the weaknesses.  And find ways
to use your emotions to engage with the problem,
without letting your reactions interfere with your ability
to work effectively.

Learning to debug can be frustrating, but it is a valuable skill
that is useful for many activities beyond programming.  At the
end of each chapter there is a section, like this one,
with my suggestions for debugging.  I hope they help!


\section{Glossary}

\begin{description}

\item[problem solving:]  The process of formulating a problem, finding
a solution, and expressing it.
\index{problem solving}

\item[high-level language:]  A programming language like Python that
is designed to be easy for humans to read and write.
\index{high-level language}

\item[low-level language:]  A programming language that is designed
to be easy for a computer to run; also called ``machine language'' or
``assembly language''.
\index{low-level language}

\item[portability:]  A property of a program that can run on more
than one kind of computer.
\index{portability}

\item[interpreter:]  A program that reads another program and executes
it
\index{interpret}

\item[prompt:] Characters displayed by the interpreter to indicate
that it is ready to take input from the user.
\index{prompt}

\item[program:] A set of instructions that specifies a computation.
\index{program}

\item[print statement:]  An instruction that causes the Python
interpreter to display a value on the screen.
\index{print statement}
\index{statement!print}

\item[operator:]  A special symbol that represents a simple computation like
addition, multiplication, or string concatenation.
\index{operator}

\item[value:]  One of the basic units of data, like a number or string, 
that a program manipulates.
\index{value}

\item[type:] A category of values.  The types we have seen so far
are integers (type {\tt int}), floating-point numbers (type {\tt
float}), and strings (type {\tt str}).
\index{type}

\item[integer:] A type that represents whole numbers.
\index{integer}

\item[floating-point:] A type that represents numbers with fractional
parts.
\index{floating-point}

\item[string:] A type that represents sequences of characters.
\index{string}

\item[natural language:]  Any one of the languages that people speak that
evolved naturally.
\index{natural language}

\item[formal language:]  Any one of the languages that people have designed
for specific purposes, such as representing mathematical ideas or
computer programs; all programming languages are formal languages.
\index{formal language}

\item[token:]  One of the basic elements of the syntactic structure of
a program, analogous to a word in a natural language.
\index{token}

\item[syntax:] The rules that govern the structure of a program.
\index{syntax}

\item[parse:] To examine a program and analyze the syntactic structure.
\index{parse}

\item[bug:] An error in a program.
\index{bug}

\item[debugging:] The process of finding and correcting bugs.
\index{debugging}

\end{description}


\section{Exercises}

\begin{exercise}

It is a good idea to read this book in front of a computer so you can
try out the examples as you go.

Whenever you are experimenting with a new feature, you should try
to make mistakes.  For example, in the ``Hello, world!'' program,
what happens if you leave out one of the quotation marks?  What
if you leave out both?  What if you spell {\tt print} wrong?
\index{error message}

This kind of experiment helps you remember what you read; it also
helps when you are programming, because you get to know what the error
messages mean.  It is better to make mistakes now and on purpose than
later and accidentally.

\begin{enumerate}

\item In a print statement, what happens if you leave out one
of the parentheses, or both?

\item If you are trying to print a string, what happens if you
leave out one of the quotation marks, or both?

\item You can use a minus sign to make a negative number like
{\tt -2}.  What happens if you put a plus sign before a number?
What about {\tt 2++2}?

\item In math notation, leading zeros are ok, as in {\tt 02}.
What happens if you try this in Python?

\item What happens if you have two values with no operator
between them?

\end{enumerate}

\end{exercise}



\begin{exercise}

Start the Python interpreter and use it as a calculator.

\begin{enumerate}

\item How many seconds are there in 42 minutes 42 seconds?

\item How many miles are there in 10 kilometers?  Hint: there are 1.61
  kilometers in a mile.

\item If you run a 10 kilometer race in 42 minutes 42 seconds, what is
  your average pace (time per mile in minutes and seconds)?  What is
  your average speed in miles per hour?

\index{calculator}
\index{running pace}

\end{enumerate}

\end{exercise}
